%!TEX TS-program = xelatex
\documentclass{nihbiosketch}

%\usepackage{draftwatermark}  % delete this in your document!
%\SetWatermarkText{Sample}    % delete this in your document!
%\SetWatermarkLightness{0.9}  % delete this in your document!

%------------------------------------------------------------------------------

\name{Naomi J. Goodrich-Hunsaker}
\eracommons{NAOMIHUNSAKER}
\position{Research Associate, Department of Neurology, University of Utah}

\begin{document}
%------------------------------------------------------------------------------

\begin{education}
	University of Utah & B.S & 05/2005 & Psychology \\
	Brigham Young University & Ph.D. & 04/2009  & Neuroscience \\
	Interdisciplinary Training Program in Neurotherapeutics  & NIH Training  & 05/2010  & \\
	University of California Davis, Davis, CA  & Postdoc  & 07/2014  & \\
\end{education}

\section{Personal Statement}

As a neuroimaging analyst, I have extensive clinical and research experience with specific training and expertise in advanced neuroimaging techniques and have applied these skills across many different types of clinical populations, including pediatric traumatic brain injury, sports concussion, autism, 22q11.2 Deletion Syndrome, Fragile X syndrome, and Alzheimer's disease. My work involves evaluating and developing novel pipelines and analytic neuroimaging methods with the goal of integrating these outcome variables into larger multidisciplinary datasets. Specifically, my expertise is in advanced neuroimaging analysis techniques (diffusion and perfusion imaging as well as volumetric analyses, functional connectivity, FLAIR, etc.) using high-performance computing (HPC) systems. However, my research has always focused on how information and mental experiences are coded and represented throughout the brain, specifically by assessing the patterns of cognitive strengths and weaknesses that scale with disease / clinical severity or dosage of a genetic mutation. I have completed research for several NIH-funded grants and successfully collaborated within a number of large consortia. In this project, I will specifically provide support for neuroimaging analyses.

%------------------------------------------------------------------------------
\section{Positions and Honors}

\subsection*{Positions and Employment}
\begin{datetbl}
	2002--2005  & Research Assistant, Department of Psychology, University of Utah, Salt Lake City, UT \\
	2007--2009  & Research Staff, Department of Critical Care \& Medicine, LDS Hospital, SLC, UT \\
	2008--2009  & Research Staff, Department of Critical Care \& Medicine, Intermountain Medical Center, Murray, UT \\
	2009--2015  & Postdoctoral Scholar, MIND Institute, University of California, Sacramento, CA \\
	2013        & Adjunct Faculty, Department of Psychology, California State University Sacramento, CA \\
	2014        & Adjunct Faculty, Utah Valley University, Orem, UT \\
	2014--Present & Adjunct Faculty, Department of Psychology, Brigham Young University, Provo, UT \\
	2016        & Independent Consultant, University of Missouri - St. Louis, St. Louis, MO \\
	2017        & Independent Consultant, Baylor College of Medicine, Houston, TX \\
	2018--Present & Research Associate, University of Utah, Salt Lake City, UT \\
\end{datetbl}

\subsection*{Skills and Knowledge Base}
\textbf{Operating systems:} Mac OS (10.x) and Linux (Red Hat, CentOS, Ubuntu); 
\textbf{Programming Languages:} Unix shell (BASH), Z shell (ZSH), Hypertext Markup Language (HTML), MATLAB, R, Python, LaTeX, Apple Script, Git, Markdown, reStructuredText, CSS, SPSS;
\textbf{High Performance Computer (HPC) batch scheduling:} SLURM, qsub;
\textbf{Biomedical Imaging Software:} acpcdetect, AFQ, ANTs, BIDS apps, BrainNet Viewer, dcm2niix, DICOM ToolKit (DCMTK), Docker, FreeSurfer, Freeview, FSL, ITKSnap, mango, Mindboggle, MRIQC, MRtrix3, OsiriX, SPM, Tracula, VistaSoft;
\textbf{R Imaging Tools:} neurobase, oro.nifti, ggseg, ANTsR, ANTsRCore, TractoR, spm12r, matlabr, freesurfer, fslr;
\textbf{Productivity Software:} ImageMagick, Google Docs (Docs, Sheets, Slides, Forms), MS Office (Word, Excel, Powerpoint), Apple iWork (Pages, Keynote, Numbers), Adobe (InDesign, Illustrator, Photoshop);
\textbf{File Synchronization:} Transmit, FileZilla, rsync;
\textbf{Statistical Software:} SPSS, R, R Studio, Jamovi;
\textbf{Publishing Books and Technical Documents}: Bookdown, Github Pages, Hugo, Jekyll

%------------------------------------------------------------------------------

\section{Contributions to Science}

\begin{enumerate}
	
	\item As a translational researcher, my contributions to science have involved a wide range of topics, but always with a central theme of trying to understand the attributes or domains of cognitive function in clinical populations and in animal models of disease. My early publications were focused on the mechanisms underlying spatial cognition. These publications found that the subregions of the hippocampus and parietal cortex process spatial information differently. 
	      
	      \begin{enumerate}
	      	
	      	\item Goodrich-Hunsaker NJ, Howard BP, Hunsaker MR, Kesner RP. (2008) Human topological task adapted for rats: Spatial information processes of the parietal cortex. Neurobiol Learn Mem, 90(2), 389--94. PMID: \href{https:/pubmed.gov/18571941}{18571941}; PMCID: \href{https://www.ncbi.nlm.nih.gov/pmc/articles/PMC2570225}{PMC2570225}.
	      	      
	      	\item Goodrich-Hunsaker NJ, Hunsaker MR, Kesner RP. The interactions and dissociations of the dorsal hippocampus subregions: how the dentate gyrus, CA3, and CA1 process spatial information. (2008). Behav Neurosci, 122(1), 16--26. PMID: \href{https:/pubmed.gov/18298245}{18298245}.
	      	      
	      	\item Goodrich-Hunsaker NJ, Hunsaker MR, Kesner RP. Dissociating the role of the parietal cortex and dorsal hippocampus for spatial information processing. (2005). Behav Neurosci, 119(5), 1307--15. PMID: \href{https:/pubmed.gov/16300437}{16300437}.
	      	      
	      \end{enumerate}
	      
	      
	\item Driven by my interests to understand the continuity of cognitive processes from rodents to humans, these studies document the performance of adult amnesic participants with focal, bilateral hippocampal damage on several well-known animal behavior paradigms. These studies emphasized that if similar methodologies were used to assess hippocampal function in both rodents and humans, then we would be able to answer the question whether the hippocampus in rodents and humans were necessary for the same mnemonic processes. This body of work when merged with the findings from previous research provide greater insight into memory processes and their neural substrates.    
	      
	      \begin{enumerate}
	      	
	      	\item Goodrich-Hunsaker NJ, Hopkins RO. Spatial memory deficits in a virtual radial arm maze in amnesic participants with hippocampal damage. (2010) Behav Neurosci, 124(3), 405--13. PMID: \href{https:/pubmed.gov/20528085}{20528085}. 
	      	      
	      	\item Goodrich-Hunsaker NJ, Livingstone SA, Skelton RW, Hopkins RO. Spatial deficits in a virtual water maze in amnesic participants with hippocampal damage. (2010). Hippocampus, 20(4), 481--91. PMID: \href{https:/pubmed.gov/19554566}{19554566}. 
	      	      
	      	\item Goodrich-Hunsaker NJ, Gilbert PE, Hopkins RO. The role of the human hippocampus in odor-place associative memory. (2009). Chem Senses, 34(6), 513--21. PMID: \href{https:/pubmed.gov/19477953}{19477953}.
	      	
	      	\item Goodrich-Hunsaker NJ, Hopkins RO. Word memory test performance in amnesic patients with hippocampal damage. (2009) Neuropsychology, 23(4), 529--34. PMID: \href{https:/pubmed.gov/19586216}{19586216}. 
	      	      
	      \end{enumerate}
	      
	\item With a team of collaborators, we directly documented whether or not dysfunction in neural circuits for spatiotemporal processing and attention are present in children and adults with the fragile X premutation allele (55-200 CGG repeats). This body of work was “game changing” because it suggested the fragile X premutation allele impacts lifespan neurodevelopmental and is not just a risk factor of neurodegeneration.
	      
	      \begin{enumerate}   
	      	
	      	\item Goodrich-Hunsaker NJ, Wong LM, McLennan Y, Tassone F, Harvey D, Rivera SM, Simon TJ. Enhanced manual and oral motor reaction time in young adult female fragile X premutation carriers. (2011). J Int Neuropsychol Soc, 17(4), 746--50. PMID: \href{https:/pubmed.gov/21554789}{21554789}; PMCID: \href{https://www.ncbi.nlm.nih.gov/pmc/articles/PMC3210929}{PMC3210929}. 
	      	      
	      	\item Goodrich-Hunsaker NJ, Wong LM, McLennan Y, Srivastava S, Tassone F, Harvey D, Rivera SM, Simon TJ. Young adult female fragile X premutation carriers show age- and genetically-modulated cognitive impairments. (2011). Brain Cogn, 75(3), 255--60. PMID: \href{https:/pubmed.gov/21295394}{21295394}; PMCID: \href{https://www.ncbi.nlm.nih.gov/pmc/articles/PMC3050049}{PMC3050049}. 
	      	      
	      	\item Goodrich-Hunsaker NJ, Wong LM, McLennan Y, Tassone F, Harvey D, Rivera SM, Simon TJ. Adult Female Fragile X Premutation Carriers Exhibit Age- and CGG Repeat Length-Related Impairments on an Attentionally Based Enumeration Task. (2011). Front Hum Neurosci, 5, 63. PMID: \href{https:/pubmed.gov/21808616}{21808616}; PMCID: \href{https://www.ncbi.nlm.nih.gov/pmc/articles/PMC3139190}{PMC3139190}.
	      	      
	      \end{enumerate}
	     
    \item Continuing my interest to assess the attributes or domains of memory function in clinical populations, this body of work focuses on the functional implications of altered hippocampus and hippocampal pathway integrity in children with chromosome 22q11.2 deletion syndrome.
	      
	      \begin{enumerate}   
	      	
	      	\item Villalón-Reina JE, Martínez K, Qu X, Ching CRK, Nir TM, Kothapalli D, Corbin C, Sun D, Lin A, Forsyth JK, Kushan L, Vajdi A, Jalbrzikowski M, Hansen L, Jonas RK, van Amelsvoort T, Bakker G, Kates WR, Antshel KM, Fremont W, Campbell LE, McCabe KL, Daly E, Gudbrandsen M, Murphy CM, Murphy D, Craig M, Emanuel B, McDonald-McGinn DM, Vorstman JAS, Fiksinski AM, Koops S, Ruparel K, Roalf D, Gur RE, Eric Schmitt J, Simon TJ, Goodrich-Hunsaker NJ, Durdle CA, Doherty JL, Cunningham AC, van den Bree M, Linden DEJ, Owen M, Moss H, Kelly S, Donohoe G, Murphy KC, Arango C, Jahanshad N, Thompson PM, Bearden CE. Altered white matter microstructure in 22q11.2 deletion syndrome: a multisite diffusion tensor imaging study. (2019). Mol Psychiatry. PMID: \href{https://www.pubmed.gov/31358905}{31358905}.
	      	
	      	\item Sun D, Ching CRK, Lin A, Forsyth JK, Kushan L, Vajdi A, Jalbrzikowski M, Hansen L, Villalon-Reina JE, Qu X, Jonas RK, van Amelsvoort T, Bakker G, Kates WR, Antshel KM, Fremont W, Campbell LE, McCabe KL, Daly E, Gudbrandsen M, Murphy CM, Murphy D, Craig M, Vorstman J, Fiksinski A, Koops S, Ruparel K, Roalf DR, Gur RE, Schmitt JE, Simon TJ, Goodrich-Hunsaker NJ, Durdle CA, Bassett AS, Chow EWC, Butcher NJ, Vila-Rodriguez F, Doherty J, Cunningham A, van den Bree MBM, Linden DEJ, Moss H, Owen MJ, Murphy KC, McDonald-McGinn DM, Emanuel B, van Erp TGM, Turner JA, Thompson PM, Bearden CE. Large-scale mapping of cortical alterations in 22q11.2 deletion syndrome: Convergence with idiopathic psychosis and effects of deletion size. (2019). Mol Psychiatry. PMID: \href{https:/pubmed.gov/29895892}{29895892}; PMCID: \href{https://www.ncbi.nlm.nih.gov/pmc/articles/PMC6292748}{PMC6292748}.
	      	
	      	\item Zhan L, Jenkins LM, Zhang A, Conte G, Forbes A, Harvey D, Angkustsiri K, Goodrich-Hunsaker NJ, Durdle C, Lee A, Schumann C, Carmichael O, Kalish K, Leow AD, Simon TJ. (2018). Baseline connectome modular abnormalities in the childhood phase of a longitudinal study on individuals with chromosome 22q11.2 deletion syndrome. Human Brain Mapp, 39(1), 232--48. PMID: \href{https:/pubmed.gov/28990258}{28990258}; PMCID: \href{https://www.ncbi.nlm.nih.gov/pmc/articles/PMC5757536}{PMC5757536}.
	      	
	      	\item Scott JA, Goodrich-Hunsaker N, Kalish K, Lee A, Hunsaker MR, Schumann CM, Carmichael OT, Simon TJ. The hippocampi of children with chromosome 22q11.2 deletion syndrome have localized anterior alterations that predict severity of anxiety. (2016). J Psychiatry Neurosci, 41(31), 203--13. PMID: \href{https:/pubmed.gov/26599134}{26599134}; PMCID: \href{https://www.ncbi.nlm.nih.gov/pmc/articles/PMC4853211}{PMC4853211}.
	      	
	      	\item Deng Y, Goodrich-Hunsaker NJ, Cabaral M, Amaral DG, Buonocore MH, Harvey D, Kalish K, Carmichael OT, Schumann CM, Lee A, Dougherty RF, Perry LM, Wandell BA, Simon TJ. Disrupted fornix integrity in children with chromosome 22q11.2 deletion syndrome. (2015). Psychiatry Res, 232(1), 106--14. PMID: \href{https:/pubmed.gov/25748884}{25748884}; PMCID: \href{https://www.ncbi.nlm.nih.gov/pmc/articles/PMC4404209}{PMC4404209}.
	      	      
	      \end{enumerate}   

    \item After learning novel neuroimaging techniques for measuring hippocampal shape and fornix integrity, I was to able apply these and other state-of-the-art neuroimaging analysis methods to other populations. These studies characterize the neural and clinical changes after brain injury (traumatic brain injury, sports-related concussion, etc.).
	      
	      \begin{enumerate}   
	      	
	      	\item Wilde EA, Newsome M, Ott SD, Hunter JV MD, Dash PK, Redell JB, Spruiell M, Diaz M, Chu ZD, Goodrich-Hunsaker NJ, Petrie JA, Li R, Levin H. Persistent Disruption of Brain Connectivity After Sports-Related Concussion in a Female Athlete. (2019). J Neurotrauma,  36(22), 3164--71. PMID: \href{https:/pubmed.gov/31119974}{31119974}; PMCID: \href{https://www.ncbi.nlm.nih.gov/pmc/articles/PMC6818484}{PMC6818484 }
	      	      
	      	\item Bigler ED, Finuf C, Abildskov TJ, Goodrich-Hunsaker NJ, Petrie JA, Wood DM, Hesselink JR, Wilde EA, Max JE. Cortical thickness in pediatric mild traumatic brain injury including sports-related concussion. (2018). Int J Psychophysiol, 132(Pt A), 99--104. PMID: \href{https:/pubmed.gov/30040986}{30040986}
	      	      
	      	\item  Goodrich-Hunsaker NJ, Abildskov TJ, Black G, Bigler ED, Cohen DM, Mihalov LK, Bangert BA, Taylor HG, Yeates KO. Age- and sex-related effects in children with mild traumatic brain injury on diffusion magnetic resonance imaging properties: A comparison of voxelwise and tractography methods. (2018). J Neurosci Res, 96(4), 626--41. PMID: \href{https:/pubmed.gov/28984377}{28984377}; PMCID: \href{https://www.ncbi.nlm.nih.gov/pmc/articles/PMC5803411}{PMC5803411}
	      	
	      	\item Bigler ED, Zielinski BA, Goodrich-Hunsaker N, Black GM, Huff BS, Christiansen Z, Wood DM, Abildskov TJ, Dennis M, Taylor HG, Rubin K, Vannatta K, Gerhardt CA, Stancin T, Yeates KO. The Relation of Focal Lesions to Cortical Thickness in Pediatric Traumatic Brain Injury. (2016). J Child Neurol, 31(11), 1302-11. PMID: \href{https:/pubmed.gov/27342577}{27342577}; PMCID: \href{https://www.ncbi.nlm.nih.gov/pmc/articles/PMC5525324}{PMC5525324}
	      	      
	      \end{enumerate} 
\end{enumerate}

\subsection*{Complete List of Published Work in MyBibliography:} 
\url{https://www.ncbi.nlm.nih.gov/myncbi/naomi.hunsaker.1/bibliography/public/}


%------------------------------------------------------------------------------

\section{Research Support}

% \subsection*{Ongoing Research Support}

% \grantinfo{R01 DA942367}{Hunt (PI)}{09/01/08--08/31/16}
% {Health trajectories and behavioral interventions among older substance abusers}
% {The goal of this study is to compare the effects of two substance abuse interventions on health 
% outcomes in an urban population of older opiate addicts.}
% {Role: Co-investigator}

% \bigskip

%\grantinfo{R01 MH922731}{Merryle (PI)}{12/15/07--11/30/15}
%{Physical disability, depression and substance abuse in the elderly}
%{The goal of this study is to identify disability and depression trajectories and demographic factors 
% associated with substance abuse in an independently-living elderly population.}
% {Role: Co-investigator}

% \bigskip

%------------------------------------------------------------------------------

\subsection*{Completed Research Support}

\grantinfo{R01 HD076885}{KO Yeates (PI)}{8/23/13--5/31/18}
{Predicting Outcomes in Pediatric Mild Traumatic Brain Injury}
{The overall goal of the proposed project is to influence clinical care and extend scientific knowledge of mild TBI in children and adolescents by examining the utility of diagnostic methods commonly used in clinical settings in the prediction of persistent post concussive symptoms (PCS) and functional impairments.}
{Role: Co-investigator}

\end{document}